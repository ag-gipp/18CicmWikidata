\documentclass[a4paper]{article}
\usepackage{onecolceurws}


\usepackage[utf8]{inputenc}
\usepackage{hyperref}

\usepackage{graphicx}
\graphicspath{{images/}}


\usepackage{color}

\definecolor{gray}{rgb}{0.4,0.4,0.4}
\definecolor{darkblue}{rgb}{0.0,0.0,0.6}
\definecolor{cyan}{rgb}{0.0,0.6,0.6}
\definecolor{darkgreen}{rgb}{0,0.5,0}

\usepackage{listings}
\usepackage{lstlinebgrd}
\usepackage{graphics}

\lstset{
  extendedchars=true,
%  basicstyle=\scriptsize,
  basicstyle=\ttfamily,
  showstringspaces=false,
  showspaces=false,
  numbers=left,
  numberstyle=\scriptsize,
  numbersep=7pt,
  tabsize=2,
  breaklines=true,
  showtabs=false,
  frame=single,
  captionpos=b,
  aboveskip=12pt,
  belowskip=12pt
}

\lstdefinelanguage{XML}
{
  morecomment=[f][\color{red}]{-\ },         % deleted lines
  morestring=[b]",
  morestring=[s]{>}{<},
  morecomment=[s]{<?}{?>},
  morecomment=[s]{!--}{--},
  morecomment=[s][\color{red}]{\$}{\$},
  commentstyle=\color{darkgreen},
  stringstyle=\color{black},
  identifierstyle=\color{darkblue},
  keywordstyle=\color{cyan},
  %basicstyle=\footnotesize,
  morecomment=[f][\color{green}]+,       % added lines
  morekeywords={xmlns,version,type}% list your attributes here
}


\usepackage[
 backend=biber,
 style=numeric,
 url=false,
 isbn=false,
 doi=false,
 bibencoding=utf8
    ]{biblatex}
\addbibresource{main.bib}

\usepackage[
  citations=true,
  hybrid=true
   ]{markdown}
\markdownSetup{
  renderers = {
    link = {\href{#2}{#1}}
  }
}
\title{Generating OpenMath Content Dictionaries from Wikidata}
\author{Moritz Schubotz \\
 Dept.~of Computer and Information Science,\\
 University of Konstanz, Box 76, 78464 Konstanz, Germany,\\
 moritz.schubotz@uni-konstanz.de
}
\institution{}
\begin{document}


\maketitle

\begin{abstract}
OpenMath content dictionaries are collections of mathematical symbols.
Traditionally, content dictionaries are handcrafted by experts.
The OpenMath specification requires a name and a textual description in English for each symbol in a dictionary.
In our recently published MathML benchmark (MathMLBen), we represent mathematical formulae in Content MathML referring to Wikidata as the knowledge base for the grounding of the semantics.
Based on this benchmark, we present an OpenMath content dictionary, which we generated automatically from Wikidata.
Our Wikidata content dictionary consists of 330 entries.
We used the 280 entries of the benchmark MathMLBen, as well as 50 entries that correspond to already existing items in the official OpenMath content dictionary entries.
To create these items, we proposed the Wikidata property P5610.
With this property, everyone can link OpenMath symbols and Wikidata items.
By linking Wikidata and OpenMath data, the multilingual community maintained textual descriptions, references to Wikipedia articles, external links to other knowledge bases (such as the Wolfram Functions Site) are connected to the
 expert crafted OpenMath content dictionaries.
Ultimately, these connections form a new content dictionary base.
This provides multilingual background information for symbols in MathML formulae.
\end{abstract}
\markdownInput{main.md}
%\cite{So2006}
\begin{table}[p]
\caption{Standard OpenMath IDs, their corresponding Wikidata labels, and the Wikidata `instance` of' relation.}
\vspace{2em}
\label{tb1}
\begin{tabular}{p{.23\linewidth}cp{.5\linewidth}}
\textbf{Wikidata Label} & \textbf{OpenMath ID} & \textbf{Instance of} \\
\href{https://www.wikidata.org/entity/Q120812}{absolute value} &\texttt{ arith1\#abs }& piecewise function, even function, idempotent function \\
\href{https://www.wikidata.org/entity/Q1226939}{division} &\texttt{ arith1\#divide }& binary operation \\
\href{https://www.wikidata.org/entity/Q131752}{greatest common divisor} &\texttt{ arith1\#gcd }& function \\
\href{https://www.wikidata.org/entity/Q40754}{subtraction} &\texttt{ arith1\#minus }& binary operation, operation \\
\href{https://www.wikidata.org/entity/Q32043}{addition} &\texttt{ arith1\#plus }& binary operation \\
\href{https://www.wikidata.org/entity/Q33456}{exponentiation} &\texttt{ arith1\#power }& operation \\
\href{https://www.wikidata.org/entity/Q601053}{nth root} &\texttt{ arith1\#root }& type of mathematical function, algebraic function \\
\href{https://www.wikidata.org/entity/Q218005}{sum} &\texttt{ arith1\#sum }& mathematical expression \\
\href{https://www.wikidata.org/entity/Q40276}{multiplication} &\texttt{ arith1\#times }& binary operation \\
\href{https://www.wikidata.org/entity/Q715358}{opposite number} &\texttt{ arith1\#unary\_minus }&  \\
\href{https://www.wikidata.org/entity/Q29175}{derivative} &\texttt{ calculus1\#diff }& unary operation, mathematical concept \\
\href{https://www.wikidata.org/entity/Q6481163}{Lambda expression} &\texttt{ fns1\#lambda }& Wikimedia disambiguation page \\
\href{https://www.wikidata.org/entity/Q244761}{function composition} &\texttt{ fns1\#left\_compose }& operator, operation \\
\href{https://www.wikidata.org/entity/Q190573}{gamma function} &\texttt{ hypergeo0\#gamma }& function \\
\href{https://www.wikidata.org/entity/Q120976}{factorial} &\texttt{ integer1\#factorial }& function \\
\href{https://www.wikidata.org/entity/Q40548497}{range} &\texttt{ interval1\#interval\_oo }& part \\
\href{https://www.wikidata.org/entity/Q177239}{limit} &\texttt{ limit1\#limit }& mathematical concept \\
\href{https://www.wikidata.org/entity/Q204}{0} &\texttt{ limit1\#null }& integer, Fibonacci number, triangular number, automorphic number, even number, non-negative integer, 0 number class, non-positive integer \\
\href{https://www.wikidata.org/entity/Q178546}{determinant} &\texttt{ linalg1\#determinant }& invariant \\
\href{https://www.wikidata.org/entity/Q44337}{matrix} &\texttt{ linalg2\#matrix }& array data structure, tensor \\
\href{https://www.wikidata.org/entity/Q2916003}{row vector} &\texttt{ linalg2\#matrixrow }& row and column vectors \\
\href{https://www.wikidata.org/entity/Q13471665}{vector} &\texttt{ linalg2\#vector }& tensor \\
\href{https://www.wikidata.org/entity/Q12139612}{list} &\texttt{ list1\#list }& creative work \\
\href{https://www.wikidata.org/entity/Q191081}{logical conjunction} &\texttt{ logic1\#and }& logical connective, boolean function \\
\href{https://www.wikidata.org/entity/Q130998}{equivalence relation} &\texttt{ logic1\#equivalent }& transitive relation, symmetric relation, reflexive relation \\
\href{https://www.wikidata.org/entity/Q82435}{e} &\texttt{ nums1\#e }& real number, transcendental number, irrational number \\
\href{https://www.wikidata.org/entity/Q193796}{imaginary unit} &\texttt{ nums1\#i }& square root, mathematical constant, Gaussian integer, imaginary number \\
\href{https://www.wikidata.org/entity/Q205}{infinity} &\texttt{ nums1\#infinity }& mathematical concept \\
\href{https://www.wikidata.org/entity/Q167}{pi} &\texttt{ nums1\#pi }& real number, transcendental number, mathematical constant, irrational number \\
\href{https://www.wikidata.org/entity/Q27058}{approximation} &\texttt{ relation1\#approx }& relation, estimation \\
\href{https://www.wikidata.org/entity/Q842346}{equality} &\texttt{ relation1\#eq }& equivalence relation, partial order \\
\href{https://www.wikidata.org/entity/Q55935291}{greater or equal to} &\texttt{ relation1\#geq }& inequation \\
\href{https://www.wikidata.org/entity/Q47035128}{greater than} &\texttt{ relation1\#gt }& inequation \\
\href{https://www.wikidata.org/entity/Q55935272}{less or equal to} &\texttt{ relation1\#leq }& inequation \\
\href{https://www.wikidata.org/entity/Q52834024}{less than} &\texttt{ relation1\#lt }& inequation \\
\href{https://www.wikidata.org/entity/Q21778965}{not equals to sign} &\texttt{ relation1\#neq }& inequality sign \\
\href{https://www.wikidata.org/entity/Q177646}{subset} &\texttt{ set1\#in }& binary relation, subclass \\
\href{https://www.wikidata.org/entity/Q185837}{intersection} &\texttt{ set1\#intersect }& binary operation, set operation \\
\href{https://www.wikidata.org/entity/Q36161}{set} &\texttt{ set1\#set }& Wikidata metaclass, Wikidata metaclass, Wikidata metaclass, class (set theory), formalization, collection \\
\href{https://www.wikidata.org/entity/Q720341}{arccosine} &\texttt{ transc1\#arccos }& inverse trigonometric function, decreasing function \\
\href{https://www.wikidata.org/entity/Q2257242}{arctangent} &\texttt{ transc1\#arctan }& inverse trigonometric function, increasing function \\
\href{https://www.wikidata.org/entity/Q1256164}{cosine} &\texttt{ transc1\#cos }& trigonometric function, even function \\
\href{https://www.wikidata.org/entity/Q1253682}{hyperbolic cosine} &\texttt{ transc1\#cosh }& hyperbolic function, even function \\
\href{https://www.wikidata.org/entity/Q47306354}{natural exponential function} &\texttt{ transc1\#exp }& exponential function, type of mathematical function \\
\href{https://www.wikidata.org/entity/Q204037}{natural logarithm} &\texttt{ transc1\#ln }& type of mathematical function, logarithm \\
\href{https://www.wikidata.org/entity/Q11197}{logarithm} &\texttt{ transc1\#log }& type of mathematical function, type of mathematical function, multivalued function, elementary transcendental function \\
\href{https://www.wikidata.org/entity/Q152415}{sine} &\texttt{ transc1\#sin }& trigonometric function, odd function \\
\href{https://www.wikidata.org/entity/Q1292101}{hyperbolic sine} &\texttt{ transc1\#sinh }& hyperbolic function, odd function \\
\href{https://www.wikidata.org/entity/Q1129196}{tangent} &\texttt{ transc1\#tan }& trigonometric function \\
\href{https://www.wikidata.org/entity/Q1274703}{hyperbolic tangent} &\texttt{ transc1\#tanh }& hyperbolic function \\
\end{tabular}
\end{table}

\paragraph*{Acknowledgments} We thank the Wikimedia Foundation and Wikimedia Deutschland for providing cloud computing facilities and for providing office space for us.
This work was supported by the FITWeltweit program of the German Academic Exchange Service (DAAD) as well as the German Research Foundation (DFG grant GI-1259-1).
The author would like to thank Howard Cohl for constructive criticism of the manuscript.
%\clearpage
\printbibliography%[keyword=primary]
\end{document}

We obtained the list 330 as follows:
After converting all valid formulae in MathMLBen to strict content MathML using MathTools, we obtained 4529 symbols.
The most frequent symbols included complex and natural numbers and subclasses such as positive natural numbers.
However, the main advantage of using Wikidata is the large variety of different symbols.

XQuery lib
Role distribution
for $x in //*:Role
  group by $who := $x/string()

return element res{$who,count($x)}

http://tinyurl.com/ych2b93l

SELECT ?item ?itemLabel ?value (GROUP_CONCAT(?classLabel; SEPARATOR=", ") as ?groupLabels)  (GROUP_CONCAT(?sclassLabel; SEPARATOR=", ") as ?sgroupLabels)  WHERE {
  ?item wdt:P5610 ?value.
  optional {?item wdt:P31 ?class.}.
  optional {?item wdt:P279 ?sclass.}.
  SERVICE wikibase:label { bd:serviceParam wikibase:language "en".
                          ?class rdfs:label ?classLabel.
                                                    ?sclass rdfs:label ?sclassLabel.
                         ?item rdfs:label ?itemLabel.} .
}
group by ?item ?itemLabel ?value

ORDER BY ?value
LIMIT 1000



While the MathMLBen user interface fetches information from Wikidata in real time to support editors of the gold standard, the content dictionary can be used by third party software that has an interface to process MathML and OpenMath but is unaware of Wikidata.
