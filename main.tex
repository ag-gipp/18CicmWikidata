\documentclass[a4paper]{article}
\usepackage{onecolceurws}


\usepackage[utf8]{inputenc}
\usepackage{hyperref}

\usepackage{graphicx}
\graphicspath{{images/}}


\usepackage{color}

\definecolor{gray}{rgb}{0.4,0.4,0.4}
\definecolor{darkblue}{rgb}{0.0,0.0,0.6}
\definecolor{cyan}{rgb}{0.0,0.6,0.6}
\definecolor{darkgreen}{rgb}{0,0.5,0}

\usepackage{listings}
\usepackage{lstlinebgrd}
\usepackage{graphics}

\lstset{
  extendedchars=true,
%  basicstyle=\scriptsize,
  basicstyle=\ttfamily,
  showstringspaces=false,
  showspaces=false,
  numbers=left,
  numberstyle=\scriptsize,
  numbersep=7pt,
  tabsize=2,
  breaklines=true,
  showtabs=false,
  frame=single,
  captionpos=b,
  aboveskip=12pt,
  belowskip=12pt
}

\lstdefinelanguage{XML}
{
  morecomment=[f][\color{red}]{-\ },         % deleted lines
  morestring=[b]",
  morestring=[s]{>}{<},
  morecomment=[s]{<?}{?>},
  morecomment=[s]{!--}{--},
  morecomment=[s][\color{red}]{\$}{\$},
  commentstyle=\color{darkgreen},
  stringstyle=\color{black},
  identifierstyle=\color{darkblue},
  keywordstyle=\color{cyan},
  %basicstyle=\footnotesize,
  morecomment=[f][\color{green}]+,       % added lines
  morekeywords={xmlns,version,type}% list your attributes here
}


%\usepackage[
% backend=biber,
% style=numeric,
% url=false,
% isbn=false,
% doi=false,
% bibencoding=utf8
%    ]{biblatex}
%\addbibresource{main.bib}

\usepackage[
  citations=true,
  hybrid=true
   ]{markdown}
\markdownSetup{
  renderers = {
    link = {\href{#2}{#1}}
  }
}
\title{Representing Mathematical Formulae in Content MathML using Wikidata}
\author{Moritz Schubotz \\
 Dept.~of Computer and Information Science,\\
 University of Konstanz, Box 76, 78464 Konstanz, Germany,\\
 moritz.schubotz@uni-konstanz.de
}
\institution{}
\begin{document}


\maketitle

\begin{abstract}
We represent mathematical formulae in Content MathML referring to Wikidata as the knowledge base for the grounding of the semantics.
By doing so, we link items in the Wikidata knowledge base to mathematical identifiers or operators.
In contrast to other mathematical knowledge bases which define symbols in a deductive fashion, the terms in Wikidata emerged inductively from Wikipedia articles in different languages.
To maximize the interoperability with non-strict content MathML, we create a Wikidata based content dictionary group.
\end{abstract}
\markdownInput{main.md}
\paragraph*{Acknowledgments} We thank Wikimedia Foundation and Wikimedia Deutschland providing cloud computing facilities and for providing office space for us.
This work was supported by the FITWeltweit program of the German Academic Exchange Service (DAAD) as well as the German Research Foundation (DFG grant GI-1259-1).
%\printbibliography[keyword=primary]
\end{document}
